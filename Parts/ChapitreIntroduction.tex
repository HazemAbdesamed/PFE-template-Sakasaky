\renewcommand\labelitemi{$\bullet$}
\renewcommand\labelitemii{$\circ$}
\chapter{Introduction générale}

%La population mondiale connaît une très forte croissance depuis 1900. Il y avait 1,5 milliard d'hommes au début du siècle, 2,5 milliards en 1950, et aujourd’hui en 2018 : 7,4 milliards. Les villes se sont développées, au cours des 19e et 20e siècles, sur des carrefours de communication, des fleuves ou sur des littoraux. Des banlieues se sont également développées autour des grandes villes. Avec ce développement, un réseau de transport public a été nécessaire pour permettre de lier les éléments essentiels de ces villes. Mais cette demande croissante de transport a engendré de nouveaux problèmes. Parmi eux, le fait que les usagers nécessitent souvent un moyen pour s’orienter et se renseigner.

L’informatique a apporté plusieurs solutions à ce problème. Il existe en effet plusieurs moyens de visualisation cartographique. On cite OpenStreetMap et Google Maps comme exemple, qui permettent la manipulation et visualisation de données géographiques du monde entier, de tracer des itinéraires entre deux points et d’aider les utilisateurs à marquer et trouver leurs magasins, hôtel ou un autre lieu favori. Cependant, ces outils ne proposent généralement qu’un seul moyen de transport. De plus, certaines de ces fonctionnalités ne sont pas adaptées pour tous les pays, voire non disponibles. Ajouté à ça un nombre immense de lignes de bus, par exemple, avec des noms très peu significatifs et une organisation aléatoire des lignes qui changent fréquemment : des facteurs contribuant encore plus à la confusion des usagers.

Ces raisons font que trouver un chemin optimal faisant bon usage de différents moyens de transport pour une personne est une tâche bien difficile, car ça demande principalement une connaissance parfaite de toute la ville, chose qui est impossible notamment pour un touriste.
Le choix du chemin dépend aussi de plusieurs facteurs : la situation et les préférences de chaque personne, ainsi que le jour du trajet, l'heure et la météo.
Par exemple, un usager peut choisir habituellement de marcher vers un arrêt de tramway et puis le prendre jusqu'à sa destination, mais préfère, en jour de pluie ou suite à défaut de pouvoir marcher, de prendre deux bus avec un chemin plus long.

Notre but est donc de créer un outil qui aide à cette décision, qui sera ensuite adapté à la ville d’Oran.

\section{Intérêt et avantages}
Offrir un guide de transport évolué a pour apport :
\begin{itemize}
	\item Gain considérable de temps, de transports et réduction de dépenses.
	\item Assurer un meilleur respect d'horaire et d'itinéraires de la part des entreprises de transport, et ainsi avoir plus de confiance de la part des utilisateurs.
	\item Encourager plus de citoyens à utiliser les transports communs.
	\item Améliorer la circulation en ville en diminuant le nombre d'utilisateurs de véhicules personnels et ainsi réduire les embouteillages.
	\item Meilleure expérience aux touristes et visiteurs.
\end{itemize}

\section{Problématique}

La création d'un outil qui aide à cette décision présente plusieurs problématiques, avant de commencer le projet, nous sommes amenés à répondre aux questions suivantes :
\begin{itemize}
	\item Quels sont ces différents facteurs et critères à considérer lors de ce choix ?
	\item Quels sont les différents usagers des transports communs, et quel est le besoin de chaque profile ?
	\item Quelles sont les différentes difficultés et contraintes qui peuvent affecter ce choix?
	\item Peut-on offrir une solution informatique évoluée pour assister à ce choix ? Si oui : 
	      \begin{itemize}
	      	\item Comment représenter un réseau routier d'une ville comme Oran, et contenant plusieurs moyens de transport ?
	      	\item Comment calculer un chemin sur ce réseau, et comment déterminer un chemin optimal ?
	      	\item Comment inclure les différentes préférences et circonstances de chaque utilisateur ?
	      	\item Comment présenter cette solution aux usagers de manière simple et efficace ?
	      \end{itemize}
\end{itemize}
			
\section{Applications similaires}			
\subsection{RATP}
L'application RATP a été développée 2010 dans le but d'injecter du réseau social dans le réseau souterrain. Aujourd'hui c'est le compagnon de transport pour Paris et l'ile de France.
Parmi les services qu'elle propose, on cite:
\begin{itemize}
	\item La possibilité de consulter les horaires de passages des différents moyens de transport public desservant Paris et ses environs. 
	\item Permet de retrouver les stations où les arrêts autour de l'endroit où se trouve l'utilisateur grâce à la géolocalisation.
\end{itemize}
Un avantage particulier avec RATP est le fait qu'il peut être paramétré de manière à émettre des alertes en cas de perturbations ou éventuels retards sur une ligne particulière.
	
\subsection{CityMapper}
CityMapper est une application de transports en commun qui a été lancé à Londres en 2011.
Elle propose 3 modes d'utilisation : 
\begin{itemize}
	\item Découvrir de tous les modes de transports en commun dans cette ville.
	\item Obtenir les différents moyens pour arriver à la destination souhaitée.
	\item Comparer le temps de chaque trajet.
\end{itemize}

CityMapper couvre en ce moment 36 villes dont Londres, Berlin, Tokyo, Paris et New York...etc.  Un de ses avantages est la possibilité de prévoir un trajet en avance et de le télécharger pour qu'il soit disponible hors connexion.

\section{Solutions proposée}
Au moment de rédiger ce mémoire, aucune application n'offre ce service en Algérie.

De ce fait, nous proposons de développer une application Web full-REST qui proposera les fonctionnalités suivantes : 

\begin{itemize}
	\item Calculer un chemin entre deux lieux saisis par l'utilisateur.
	\item Proposer un chemin optimal faisant usage de plusieurs moyens de transport public, tout en donnant la main à l'utilisateur pour choisir les moyens désirés.
	\item Prendre en compte la possibilité de marche dans les chemins suggérés.
	\item Se baser sur différents critères dans ce choix, nous prendrons les 04 critères suivants : 
	\begin{itemize}
		\item Minimum de temps.
		\item Minimum de marche.
		\item Minimum de correspondance.
		\item Minimum de cout.
	\end{itemize}	 
\end{itemize}

Ce projet sera composé de 04 parties :
\begin{description}
	\item[Conception et stockage des données: ] nous étudierons le problème pour enfin proposer une bonne représentation des données que nous pourrons facilement utiliser dans la recherche du chemin.

	\item[Service Web: ] en ce moment, les Services Web sont une tendance sur le Web, presque indispensable dans toute application non triviale: nous commencerons donc par créer une API REST complète de l'application afin qu'elle soit facilement extensible et portable sur différentes plateformes par la suite.
	
	\item[Interface d'administrateur: ] implémenter une application indépendante qui va faciliter la  communication avec l'API pour l'ajout et modification des lignes de transport et autres informations nécessaires à l'application.
	
	
	\item[Interface Web: ] implémenter une Application Web client afin de visualiser les fonctionnalités de l'API.
\end{description}

Vu le grand nombre de fonctionnalités et la complexité de cette problématique, nous limiterons la version initiale de l'application à un premier algorithme de recherche simple mais applicable sur la ville d'Oran.

L'architecture de l'application devra être suffisamment flexible pour pouvoir améliorer la recherche du chemin, et ajouter progressivement des améliorations ou de nouveaux critères.
	 
\section{Plan du rapport}

Après ce premier chapitre où nous avons présenté les objectifs du projet et la solution proposée; 

Nous présenterons tout au long du Chapitre 2 ce qu'est un service Web, ses caractéristiques et avantages, nous explorerons ensuite deux manières concurrentes pour implémenter des Services Web : SOAP et REST pour débattre brièvement les points forts de chacune afin de justifier notre décision d'utiliser REST.

Le chapitre 3 présentera en première partie les définitions relatives aux graphes et recherche de chemins, et traitera ensuite les différents détails concernant la représentation et stockage des données de l'application. 

( À faire après chapitre 4) \newline
Nous présenterons ensuite dans le chapitre 4 les différentes technologies qu'on a utilisé, ainsi qu'un aperçu des représentations utilisées et des détails implémentations. ***qui seront accompagnées par des captures d'écran décrivant le fonctionnement de l'application.**
Nous parleront	 enfin dans le chapitre 5 sur les différentes perspectives et futur stuff
