\chapter{Introduction générale}
	\newpage	
	\section{Motivation}
		
		** Idées à inclure 
		\begin{itemize}
		\item commencer par parler sur la société moderne et la croissance des ville
		\item parler de l'importance du transport
		\item passer aux solutions qu'a apporté l'informatique pour faciliter les besoins quotidiens, en citant Google Maps ou tout autre outil cartographie
		\item Mais ces outils n'offrent qu'un seul moyen de 
		\item N'indiquent pas quel moyen utiliser, notamment pour le transport public.
		\item Parler le grand nombre de lignes de deux entreprises différentes (publique et privé), qui offrent un grand nombre de lignes un peu aléatoires, sans guide ou noms significatifs ce qui rends le transport pour les visiteurs d'Oran
		\end{itemize}


Ces raisons font que trouver un chemin optimal faisant bon usage de différents moyens de transport pour une personne est une tâche bien difficile, car ça demande principalement une connaissance parfaite de toute la ville, chose qui est impossible notamment pour un touriste.
Le choix du chemin dépend aussi de plusieurs facteurs : la situation et les préférences de chaque personne, ainsi que le jour du trajet, l'heure et la météo.
Par exemple, un usager peut choisir habituellement de marcher vers un arrêt de tramway et puis le prendre jusqu'à sa destination, mais préfère, en jour de pluie ou suite à défaut de pouvoir marcher, de prendre deux bus avec un chemin plus long.

Notre but est donc de créer un outil qui aide à cette décision, qui sera ensuite adapté à la ville d’Oran.

	\section{Intérêt et avantages}
		Offrir un guide de transport évolué a pour apport :
		\begin{itemize}
		\item Gain considérable de temps, de transport et réduction de dépenses.
		\item Assurer un meilleur respect d'horaire et d'itinéraires de la part des entreprises de transport, et ainsi avoir plus de confiance de la part des utilisateurs.
		\item Encourager plus de citoyens à utiliser les transports communs.
		\item Améliorer la circulation en ville en diminuant le nombre d'utilisateurs de véhicules personnel et ainsi réduire les embouteillages.
		\item Meilleure expérience aux touristes et visiteurs.
		\end{itemize}

	\section{Problématique}

	La création d'un outil qui aide à la décision présente plusieurs problématiques, avant de commencer le projet, nous sommes amenés à répondre aux questions suivantes :
		\begin{itemize}
		\item Quels sont ces différents facteurs et critères à considérer lors de ce choix ?
		\item Quelles sont les différents usagers des transports communs, et quel est le besoin de chaque profile ?
		\item Quelles sont les différentes difficultés et contraintes qui peuvent affecter ce choix?
		\item Peut on offrir une solution informatique évoluée pour assister à ce choix ? Si oui : 
			\begin{itemize}
			\item Comment représenter un réseau routier d'une ville comme Oran, et contenant plusieurs moyens de transport ?
			\item Comment calculer un chemin sur ce réseau, et comment déterminer un chemin optimal ?
			\item Comment inclure les différentes préférences et circonstances de chaque utilisateur ?
			\item Comment présenter cette solution aux usagers de manière simple et efficace ?
			\end{itemize}
		\end{itemize}
			
		\section{Applications similaires}			
		\subsection{RATP}
		
		\subsection{CityMapper}
		
			
		\section{Solution proposée}
		Au moment de rédiger ce mémoire, aucune application n'offre ce service en Algérie, cependant, ** Parler de l'application Yassir ? pour dire que l'utilisation des outils de cartographie et la recherche de chemin à Oran sont fonctionnels.
		
		
		**Nous proposons de développer une application Web full-REST (**Motiver pourquoi) qui aura pour fonctionnalités principales : 
		\begin{itemize}
		\item Citer les fonctionnalités.
		\item Fixer les buts de l'application (cerner le problème)
		\item L'application Web exposera un Service Web --> permettre l'utilisation de l'application sur n'importe quelle plateforme 
		\item Développer ensuite une interface Web responsive pour utiliser l'application.
		\item Développer une interface administrateur pour faciliter la manipulation de la  base de donnée.
		\item Parler ici des objectifs du projet, representation et tout.
		\end{itemize}
		 
		 
		 \newpage
		 \section{Plan du rapport}