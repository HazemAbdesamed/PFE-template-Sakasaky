\chapter{Conclusion et perspectives}

Notre objectif était de créer un outil qui aide à trouver un chemin optimal dépendant de plusieurs facteurs et faisant bon usage de différents moyens de transport. 

Nous avons proposé  de développer une application Web full-REST qui propose plusieurs fonctionnalités, notamment le calcul de chemin entre deux lieux saisis par l'utilisateur selon différents critères.


Nous avons divisé la résolution de ce problème en 4 parties:

\begin{description}
\item[Conception et stockage des données:] 
\item[Service Web:]
\item[Interface Web:]
\item[Interface administrateur:]
\end{description}


Lors de ce projet, nous avons fais face à certaines difficultés qui ont attardées la version complète de l'application, on cite en premier lieu la spécificité de nos lignes, le manque de données et la non disponibilité des données GPS, qui a suscité d'implémenter une interface administrateur permettant de mieux organiser nos informations.\newline\newline
Ayant désormais une base solide de l'application, nous pouvons nous focaliser sur les idées et objectifs suivants au futur :
\begin{itemize}
	\item \textbf{Compléter les données:} en introduisant la liste complète des lignes et informations de transport pour avoir une application utilisable.
	
	\item \textbf{Enrichir les propriétés des stations: }en ajoutant des informations tel que :\newline
		\begin{itemize}
			\item \textbf{Disponibilité }: possibilité de rendre les stations en cas de travaux (ou autres perturbations) non actives pour quelques jours.
			\item
		\end{itemize}
		
	\item \textbf{Ajouter la fonctionnalité de recherche du plus proche arrêt: } pour permettre de trouver un chemin à partir de n'importe quel point : cette fonctionnalité nécessite d'implémenter des \emph{opérations spatiales} sur le graphe. Par ailleurs, Neo4J permet déjà d'intégrer ces opérations facilement grâce à des librairies tel que \textbf{Neo4j Spatial}, et donc il ne sera pas nécessaire d'utiliser d'autres SGBD pour y arriver.

	\item \textbf{Améliorer l'algorithme de recherche de chemin}.
	
	\item \textbf{Ajouter des fonctionnalités supplémentaires à l'application admin}.
	
	\item \textbf{Ajouter des fonctionnalités supplémentaires à l'application client,} par exemple:
		\begin{itemize}
			\item Intégrer des cartes interactives permettant de visualiser les stations parcourues en temps réel, ou de télécharger les instructions en PDF pour les utiliser ultérieurement.
			\item Intégrer dans l'interface les futures fonctionnalités : trouver la station plus proche en marquant sur carte, prendre en compte la météo et l'heure (avec éventuellement un mode nuit).
			\item Améliorer la version mobile de l'application web ou une une application Android pour plus d'interactivité.
			\item Compléter la page d'information en affichant les lignes sur carte, les informations sur les horaires et les numéros de taxis.
			\item Ajouter la possibilité de créer des comptes utilisateurs pour pouvoir enregistrer des lieux personnalisés (Maison, école,..etc.).
			\item Ajouter un espace signalisation pour permettre aux utilisateurs de poster un changement dans les lignes ou une information incorrecte afin d'assurer l'intégrité et la mise à jour des informations.
		\end{itemize}
\end{itemize}