\chapter{Représentation du réseau routier}
**
**Un graphe sert mieux à définir l'existence d'une relation entre objets tels qu'une ligne entre deux stations de métro, une relation d'amitié dans un réseau social ou encore une rue entre deux carrefours

	\section{Généralités sur les graphes}
	La théorie des graphes est très probablement née en 1735 lorsque Leonhard Euler (1707 - 1783) résout le problème des sept ponts de Königsberg. 
L’énoncé de ce problème est: La ville de Königsberg est une ville autour d'un fleuve, elle compte quatre berges et sept ponts les reliant. Le but du jeu est de savoir s'il existe un chemin permettant d'emprunter tous les ponts une fois et une seule et revenir au point de départ. Le problème s'appelle, de façon plus formelle, la recherche d'un cycle eulérien dans un graphe. Euler a démontré que ce problème n'avait pas de solution.

	\subsection{Définitions}
	\textbf{Graphe :} Un graphe est composé de sommets (\textbf{vertices}) ou noeuds (\textbf{nodes}), et d'arcs (\textbf{edges}) ou d’arêtes (\textbf{links}) reliant certains de ces sommets ou noeuds.
	Un graphe G est défini de manière formelle par un couple (S,A) où :
	\begin{itemize}
	\item S est un ensemble fini d'éléments. Chacun de ces éléments est appelé sommet du graphe.
	\item A est un sous ensemble (éventuellement nul) de SxS. Chacun de ces éléments de A est appelé arc ou arête.
	\end{itemize}
	Chaque arc est associé un poids qui le décrit. Par exemple, dans un réseau social il peut définir la nature de la relation (ami, famille, collègue) et dans un réseau routier la longueur d’une rue. Parfois le terme coût est utilisé. 
	
	\textbf{Graphe connexe: } De manière intuitive, la notion de connexité est triviale, mais importante dans notre cas. Un graphe est connexe si on peut atteindre n’importe quel sommet à partir d’un sommet quelconque en parcourant différentes arêtes.
	
	\section{Avantages d'utilisation d'un graphe}
	\subsection{Domaines d'utilisation des graphes}
	Citer certains domaines qui utilisent des graphes
	\subsection{Recherche de chemins}
	Ce qu'est un chemin, caractéristiques d'un chemin dans un graphe. Parler des algorithmes connus : Djikstra, Bellmand Ford et quelques contraintes (poids..etc)
	
	\section{Données collectées}
	\subsection{ETO (Entreprise de Transport d'Oran)}
	\begin{itemize}
	\item Nous avons été acceuilis par un des responsable de l'ETO, qui nous a fourni plusieurs informations sur ( comment marche chaque ligne, frequences, horaires, temps d'été/hiver...etc)
	\item Parler des données des lignes de l'ETO qu'il nous a fourni
	\item Parler des inconvenients ? Adresses non completes, informations mal classées,...
	\item Remerciements .
	\end{itemize}
	\subsection{DTW (Direction Transport)}
	** To be done
	\subsection{Données supplémentaires}
	Données qu'on a ajouté nous-même, parler de l'idée du crowd-sourcing.
	\subsection{Traitement de données}
	Comment on a utiliser ces données (ex : trouver coordonnées, calculer distance avec les coordonnées), et les outils/programmes écrits pour automatiser.. si possible).
	
	** Nous serons contraint d'utiliser les adresses données comme premiere version, vu l'absence de données GPS, nous (*ajouter une fonctionnalité pour pouvoir integrer aisément les coordonnées GPS au dessus des données existantes.


	\section{Construction du graphe}
	\subsection{Différente approches}
	\begin{itemize}
	\item \textbf{Outils Open Source (OpenTripPlanner} : 
		Il existe plusieurs outils qui proposent ce service, en particulier OpenTripPlanner : Un projet Open Source qui permet de créer un réseau routier à partir de données GTFS \FancyFootNote{GTFS (General Transit Feed Specification : ...}, ces données seront ensuite intégrées avec OpenStreetMap et stockée sur le serveur, qui exposera une API REST pour questionner le serveur : Recherche de Chemin, Possibilité d'intégrer les horaires, ...etc.
		
		Vu la nature un peu particulière du réseau d'Oran, et la non-disponibilité des données (Données officielles des lignes et données (adresses) sur OpenStreetMap), cette solution ne sera pas envisagée. 
		Cependant, l'application prendra une architecture flexible permettant d'intégrer, au futur, de tels outils rapidement au cas de nécessité.
	\item \textbf{Représentation indépendante de chaque ligne}
	Une des approches considérées était de représenter chaque ligne indépendamment, l'algorithme du service aura à chercher des points de liaison entre ces lignes.
	L'avantage principal de cette approche est la facilité de manipulation de ces données de lignes, en ajoutant/supprimant des lignes sans conflits.
	Cette approche présente par contre un inconvénient majeure au niveau de performance, vu que l'utilisation d'un algorithme de PathFinding requiert l'utilisation de plusieurs tables (lignes) à chaque requête. Ce qui nous a mené à l'approche suivante.
	\item \textbf{Représentation en graphe}:
	
	\begin{itemize}
	\item Representation la plus intuitive
	\item Permet un accès plus rapide, en stockant le graphe déja construit (i.e stocker dans une base de données orientée graphe)
	\item Avantages utilisation des algorithmes de recherche de chemin efficaces, meilleures performances...etc
	\item Citer les inconvenients (*Proposer les solutions et changements faits pour contrer ça) Inconvénients lors de la modification, modification d'une station peut affecter plusieurs lignes, modification d'une ligne peut poser plusieurs conflits ou imposer la reconstruction du graphe...
	\end{itemize}

	\end{itemize}
	\subsection{Représentation choisie}
	décrire la representation, ainsi que les avantages de celle ci ( ou les inconvenients des autres)
	\subsection{Résultat final}
	\subsection{Contraintes}
	Parler de quelques limitations (Espace, complexité d'ajout...etc)

				